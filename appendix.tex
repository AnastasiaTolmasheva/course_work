\newpage
\setcounter{table}{0}
\setcounter{figure}{0}
\sectionnonumber{Приложения}
\label{sec:AppendixA}
\appendixsubsubsection{Приложение А. Примеры размеченных аккаунтов}


Примеры аккаунтов, которые были размечены, представлены в таблицах ~\ref{tabular:fakeAccounts}--~\ref{tabular:realAccounts}.

\begin{table}[h]
    \caption{Фиктивный аккаунт}
    \vspace{1em}
    \label{tabular:fakeAccounts}
    \fontsize{12}
    \selectfont
    \begin{tabular}{|l|l|}
    \hline
    \multicolumn{1}{|c|}{\textbf{Характеристика}} & 
    \multicolumn{1}{c|}{\textbf{Значение}}\\ \hline
        user\_id & 6758 \\ 
        \textbf{username} & \textbf{fishzupic438} \\ 
        password & 6cd28fca7bdf8937ee45eb65d0a126a5 \\ 
        \textbf{email} & \textbf{fishgvnpo334@gmail.com} \\ 
        \textbf{url} & \textbf{https://aktivator-kleva.com} \\ 
        phone & 88298893596 \\ 
        \textbf{mailing\_address} & \textbf{fishfmmzj822@gmail.com} \\ 
        billing\_address & \texttt{NULL} \\ 
        country & MT \\ 
        locales &  \\ 
        date\_last\_email & \texttt{NULL} \\ 
        \textbf{date\_registered} & \textbf{2021-03-04 22:22:07} \\ 
        date\_validated & \texttt{NULL} \\ 
        \textbf{date\_last\_login} & \textbf{2021-03-04 22:22:07} \\ 
        must\_change\_password & 0 \\ 
        auth\_id & \texttt{NULL} \\ 
        auth\_str & \texttt{NULL} \\ 
        disabled & 0 \\ 
        disabled\_reason & \texttt{NULL} \\ 
        inline\_help & 0 \\ 
        gossip & \texttt{NULL} \\ 
        fake & 1 \\ 
        \hline
    \end{tabular}
\end{table}

\newpage
\begin{flushright}
Окончание приложения А 
\end{flushright}
\vspace{-1.5em}

\begin{table}[h]
    \caption{Настоящий аккаунт}
    \vspace{1em}
    \label{tabular:realAccounts}
    \fontsize{12}
    \selectfont
    \begin{tabular}{|l|l|}
    \hline
    \multicolumn{1}{|c|}{\textbf{Характеристика}} & 
    \multicolumn{1}{c|}{\textbf{Значение}}\\ \hline
        user\_id & 2 \\
        username & tcymblerml \\
        password & \$2y\$10\$xdvbQezhMcNsQcOf4b5JZOZ \\
        \newline & g2butksNxug7/TiEV.sClODdS4djJK \\
        email & mzym@susu.ru \\
        url & http://mzym.susu.ru/ \\
        phone &  \\
        mailing\_address &  \\
        billing\_address & \texttt{NULL} \\
        country & Ru \\
        locales &  \\
        date\_last\_email & 2020-01-03 18:42:34 \\
        date\_registered & 2014-01-24 13:57:45 \\
        date\_validated & \texttt{NULL}  \\
        date\_last\_login & 2022-12-04 08:56:49 \\
        must\_change\_password & 0 \\
        auth\_id & 1 \\
        auth\_str & \texttt{NULL}  \\
        disabled & 0 \\
        disabled\_reason & \texttt{NULL}  \\
        inline\_help & 0 \\
        gossip & \texttt{NULL}  \\
        fake & 0 \\
        \hline
    \end{tabular}
\end{table}



\newpage
\appendixsubsubsection{{Приложение Б. Спецификация вариантов использования}}


Спецификация вариантов использования (ВИ) разработанной системы приведена в таблицах~\ref{tab:UploadingData}--~\ref{tab:SearchFake}.


\begin{table}[H]
    \setstretch{1.0}
    \caption{Спецификация варианта <<Загрузить данные>>}
    \vspace{1em}
    \small
    \begin{tabular}{|p{15cm}|}
       \hline
        Прецедент: Загрузить данные\\ \hline
        ID: 1\\ \hline
        Краткое описание: \\ Исследователь загружает данные в систему.\\ \hline
        Главные актеры: \\ Исследователь.\\ \hline
        Второстепенные актеры: \\ Нет.\\ \hline
        Предусловия: отсутствуют.\\ \hline
        Основной поток:\\
        1. Прецедент начинается, когда Исследователь нажимает на кнопку <<Загрузить данные>>.\\
        2. Исследователь выбирает необходимый файл.\\
        3. Система сохраняет выбранный датасет в базе данных.
        \\ \hline
        Постусловия:\\ Исследователь загрузил датасет в базу данных. \\ \hline
        Альтернативные потоки:\\ Нет. \\ \hline
    \end{tabular}
    \label{tab:UploadingData}
\end{table}


\begin{table}[H]
    \setstretch{1.0}
    \caption{Спецификация варианта <<Редактировать данные>>}
    \vspace{1em}
    \small
    \begin{tabular}{|p{15cm}|}
       \hline
        Прецедент: Редактировать данные\\ \hline
        ID: 2\\ \hline
        Краткое описание: \\ Исследователь редактирует загруженные данные.\\ \hline
        Главные актеры: \\ Исследователь.\\ \hline
        Второстепенные актеры: \\ Нет.\\ \hline
        Предусловия: данные для редактирования были загружены.\\ \hline
        Основной поток:\\
        1. Прецедент начинается, когда Исследователь нажимает на кнопку <<Отредактировать данные>>.\\
        2. Исследователь выбирает данные, которые будет редактировать.\\
        3. Исследователь вносит изменения в датасет.\\
        4. Исследователь нажимает на кнопку <<Сохранить>>.
        \\ \hline
        Постусловия:\\ Данные были отредактированы. \\ \hline
        Альтернативные потоки:\\ Нет. \\ \hline
    \end{tabular} 
    \label{tab:Razmetka}
\end{table}
\vspace{2em}


\newpage
\begin{flushright}
Окончание приложения Б
\end{flushright}
\vspace{-1.5em}


\begin{table}[H]
    \setstretch{1.0}
    \caption{Спецификация ВИ <<Выполнить настройку>>}
    \vspace{1em}
    \small
    \begin{tabular}{|p{15cm}|}
       \hline
        Прецедент: Выполнить настройку\\ \hline
        ID: 4\\ \hline
        Краткое описание: \\ Исследователь выбирает алгоритм и настраивает его параметры. \\ \hline
        Главные актеры: \\ Исследователь.\\ \hline
        Второстепенные актеры:\\ Нет. \\ \hline
        Предусловия: \\ Загружены данные. \\ \hline
        Основной поток\\ 
        1. Прецедент начинается, когда пользователь нажимает на кнопку <<Найти фейки>>. \\ 
        2. Пользователь выбирает необходимый алгоритм и вводит его параметры. \\  \hline
        Постусловия: \\ Выполнена настройка алгоритма. \\ \hline
        Альтернативные потоки: \\ Нет \\ \hline
        \end{tabular} 
    \label{tab:Settings}
\end{table}

\begin{table}[H]
    \setstretch{1.0}
    \caption{Спецификация ВИ <<Найти фиктивные аккаунты>>}
    \vspace{1em}
    \small
    \begin{tabular}{|p{15cm}|}
       \hline
        Прецедент: Найти фиктивные аккаунты\\ \hline
        ID: 5\\ \hline
        Краткое описание: \\ Исследователь запускает алгоритм для поиска фиктивных аккаунтов. \\ \hline
        Главные актеры: \\ Исследователь.\\ \hline
        Второстепенные актеры:\\ Нет. \\ \hline
        Предусловия: \\ Загружен файл с данными, выполнена настройка алгоритма. \\ \hline
        Основной поток\\ 
        1. Прецедент начинается, когда пользователь нажимает на кнопку <<Найти фейки>> в окне настроек. \\ 
        2. Система находит фиктивные аккаунты и выводит результат работы алгоритма. \\  \hline
        Постусловия: \\ Выполнен поиск фиктивных аккаунтов. \\ \hline
        Альтернативные потоки: \\ Нет \\ \hline
        \end{tabular} 
    \label{tab:SearchFake}
\end{table}
\vspace{2em}

