\newpage
\thispagestyle{empty}

\begin{adjustwidth}{-1.5cm}{0.5cm}
\begin{linespread}{1}
\begin{center}


\small{
МИНИСТЕРСТВО ОБРАЗОВАНИЯ И НАУКИ РОССИЙСКОЙ ФЕДЕРАЦИИ\\
Федеральное государственное бюджетное образовательное учреждение\\
высшего образования\\
\textbf{<<Южно-Уральский государственный университет\\
(национальный исследовательский университет)>>\\
Высшая школа электроники и компьютерных наук\\
Кафедра системного программирования}
}



\vspace{2em}

\hfill{}
\parbox{7cm}{
УТВЕРЖДАЮ \\
Зав. кафедрой СП \\[0.5em]
\underfield{} Л.Б.~Соколинский \\[0.5em]
<<\underline{\qquad}>>\underfield{} 2024~г.}

\vspace{2.5em}

\textbf{ЗАДАНИЕ} \\
% \parbox[t]{14cm}{
\textbf{на выполнение курсовой работы}\\
по дисциплине «Программная инженерия»\\
студенту группы КЭ-303\\
Толмачевой Анастасии Вячеславовне,\\
обучающемуся по направлению\\
09.03.04 «Программная инженерия»
% }

\end{center}

\vspace{1.5em}

{
\small
\begin{enumerate}[itemsep=0cm, parsep=0cm]
	\bf\item Тема работы \rm
	\\
	Разработка системы для выявления фиктивных аккаунтов Open Journal System.

	\bf\item Срок сдачи студентом законченной работы: \rm
	31.05.2024.
        \vspace{0.2em}
	\bf\item Исходные данные к работе\rm
	\begin{enumerate}[itemindent=0.9cm, leftmargin=-0.6cm, itemsep=0cm, parsep=0cm, after=\vspace{-0.05cm}, before=\vspace{-0.05cm}]
		\raggedright
%[itemindent=2cm, leftmargin=0cm, labelsep=0.25cm, topsep=0cm, itemsep=0cm, parsep=0cm, label=\arabic*), after=\vspace{-0.2cm}]
  
		\item Open Journal Systems. [Электронный ресурс] URL: https://openjournalsystems.com/

		\item Кластеризация. [Электронный ресурс] URL: https://scikit-learn.ru/clustering/
  
            \item Han J., Kamber M., Pei Jian. Data Mining: concepts and techniques. // 3rd ed. -- Elsevier Inc, 2011. -- 703 p.

            \item Сорокин А.Б., Железняк Л.М. Технологии обучения: кластеризация и классификация: Учебное пособие – М.: РТУ МИРЭА, 2021. – 49 с.

  		\item Rokach L., Maimon O. Clustering Methods // The Data Mining and Knowledge Discovery Handbook, 2005. -- 321-352 pp. DOI: 10.1007/0-387-25465-X\_15.


	\end{enumerate}

	\bf\item Перечень подлежащих разработке вопросов\rm
	\begin{enumerate}[itemindent=0.9cm, leftmargin=-0.6cm, itemsep=0cm, parsep=0cm, before=\vspace{-0.05cm}]
		\item Анализ предметной области и литературы по теме работы.
		\item Проектирование интерфейса программной системы и модульной структуры приложения.
		\item Реализация программной системы, выявляющей фиктивные аккаунты с помощью алгоритмов машинного обучения.
            \item Подготовка набора тестов и тестирование программной системы.
	\end{enumerate}

	\bf\item Дата выдачи задания: \rm
	<<\underline{\qquad}>>\underfield{} 2024~г.
\end{enumerate}

\vspace{1em}

\noindent
{Научный руководитель}
\hfill
\hbox to 8em{М.Л.~Цымблер\hfill}

\vspace{1em}

\noindent
{Задание принял к исполнению}
\hfill
\hbox to 8em{А.В.~Толмачева\hfill}

}

\thispagestyle{empty}

\end{linespread}
\end{adjustwidth}

\pagebreak
