\newpage
\sectionnonumber{Введение}

\textbf{Актуальность темы}

Интернет в наши дни развиваются быстро. Это очень важно для маркетинговых компаний и знаменитостей, которые желают увеличить количество покупателей и фанатов; для обычных пользователей, которые быстрее узнают новости и находят новых знакомых; для исследователей, которые делятся последними открытиями и актуальной информацией. Социальные сети делают нашу жизнь лучше, но есть некоторые аспекты, которые требуют внимания.

Вместе с расширением онлайн-сообществ возникает проблема поддельных аккаунтов. <<Фейки>> могут быть созданы с разными целями: получить коммерческую выгоду, дискредитировать настоящего пользователя, заполучить личную информацию, осветить вредный контент и так далее~\cite{BoshmafMBR11}. Иногда их тяжело отличить от аккаунта живого человека, а массовость проблемы может набирать социально-опасный характер: например, <<боты>>, которые сыграли роль в выборах президента~\cite{HsuKJ19}.

Поддельные аккаунты могут встречаться и в среде научных публикаций. Open Journal System не защищена от появления в ней ненастоящих пользователей, что может вызывать различные трудности при поиске научных статей и обмене исследованиями. Существующие решения не рассматривают проблему распознавания фиктивных аккаунтов в Open Journal Systems.

Технический прогресс, особенно в области машинного обучения и анализа данных, предоставляет инструменты для обнаружения <<фейков>>. Алгоритмы машинного обучения могут быть применены для выявления аномального поведения, классификации аккаунтов как фиктивных~\cite{HassanAA23}.

Таким образом, разработка системы для выявления фиктивных аккаунтов в Open Journal System становится актуальной и востребованной, учитывая масштаб проблемы появления <<ботов>> в Интернете и технический прогресс в области искусственного интеллекта и машинного обучения.

\textbf{Цель и задачи}

Целью курсовой работы является реализация программной системы, выявляющей фиктивные аккаунты в Open Journal Systems. Для достижения поставленной цели необходимо решить следующие задачи:

\begin{enumerate}[itemindent=2cm, leftmargin=0cm, labelsep=0.4cm, topsep=0cm, itemsep=0cm, parsep=0cm, label=\arabic*., after=\vspace{-0.1cm}, before=\vspace{-0.1cm}]
	\item Провести анализ предметной области и литературы по теме работы.
	\item Спроектировать интерфейс программной системы и модульной структуры приложения.
	\item Реализовать систему, выявляющую фиктивные аккаунты с помощью алгоритмов машинного обучения.
        \item Подготовить набор тестов, выполнить тестирование программной системы.
\end{enumerate}

\textbf{Структура и содержание работы}

Курсовая работа состоит из введения, четырех разделов, заключения и списка литературы. Объем работы -- 51 страница, объем библиографического списка -- 25 наименований.

В первом разделе, «Анализ предметной области», содержится описание Open Journal System, обзор аналогичных проектов, методов поиска аномалий и классификации, определение метрик, описание наборов данных для обучения и тестирования.

Во втором разделе, <<Проектирование>> представлены варианты использования системы, функциональные и нефункциональные требования, архитектура системы, графический интерфейс и описание баз данных.

В третьем разделе, <<Реализация>>, приведены программные средства реализации, а также описаны созданные модули системы.

В четвертом разделе, <<Тестирование>>, представлено функциональное тестирование и сравнение алгоритмов.

В приложении А содержатся примеры размеченных аккаунтов.

В приложении Б представлена спецификация вариантов использования системы.
