\newpage
\section{Проектирование}
\label{sec:designing}
%В разделе 2.1. описаны варианты использования системы. В разделе 2.2. представлены функциональные и нефункциональные требования к системе. Раздел 2.3 содержит архитектуру системы. В разделе 2.4. показаны зарисовки графического интерфейса. Базы данных и их 

%\vspace{1.5em}
\subsection{Варианты использования системы}
\label{subsec:Variants}
Описание способов взаимодействия с системой, а также кто с ней может работать, отображено на рисунке~\ref{ris:variantsUse}.

\begin{figure}[ht]
    \center{\includegraphics[width=1\linewidth]{image}}
    \fbox{\includegraphics[scale=0.90]{Курсовая работа/pic/Варианты использования системы.png}}
    \caption{Диаграмма вариантов использования}
    \label{ris:variantsUse}
\end{figure}

Единственный актер, который может взаимодействовать с системой распознавания фиктивных аккаунтов в Open Journal System -- это исследователь. Он может выполнять следующие действия:

\begin{enumerate}[itemindent=2cm, leftmargin=0cm, labelsep=0.3cm, topsep=0cm, itemsep=0cm, parsep=0cm, label=\arabic*., after=\vspace{-0.1cm}, before=\vspace{-0.1cm}]
    \item Загрузить данные. Исследователь может добавить датасет в базу данных системы.
    \item Редактировать данные. Исследователь может отредактировать данные, которые загрузил.
    \item Выполнить настройку. Исследователь может выполнить настройку того, какой будет выполняться алгоритм, с какими параметрами, а также в каком формате будут выводиться результаты.
    \item Найти фиктивные аккаунты. Исследователь может найти <<фейки>> с помощью выбранного алгоритма.
\end{enumerate}


\vspace{1.5em}
\subsection{Требования к системе}
\label{subsec:fotmalDefinition}
\textbf{Функциональные требования}

Функциональные требования -- это спецификация того, каким образом должно вести себя разрабатываемое программное обеспечение. Они определяют, какие конкретные функции и возможности должны быть включены в приложение, чтобы оно соответствовало ожиданиям пользователей. Эти требования описывают конкретные действия, которые система должна выполнять, и каким образом пользователь взаимодействует с приложением для достижения определенных целей. 

Функциональные требования к системе:

\begin{enumerate}[itemindent=2cm, leftmargin=0cm, labelsep=0.3cm, topsep=0cm, itemsep=0cm, parsep=0cm, label=\arabic*., after=\vspace{-0.1cm}, before=\vspace{-0.1cm}]
    \item Система должна предоставлять пользователю возможность загружать данные для исследования.
    \item Система должна предоставлять возможность изменять загруженные данные.
    \item Система должна предоставлять пользователю возможность выполнить настройку ее работы: выбрать алгоритм, изменить его характеристики, добавить вывод результатов в графической форме.
    \item Система должна находить фиктивные аккаунты.
\end{enumerate}

\textbf{Нефункциональные требования}

Ненфункциональные требования определяют свойства и характеристики, которыми должна обладать система. Эти требования касаются аспектов, не связанных напрямую с конкретными функциями, а определяют общие качественные аспекты, которые необходимы для функционирования системы.

Нефункциональные требования к системе:

\begin{enumerate}[itemindent=2cm, leftmargin=0cm, labelsep=0.3cm, topsep=0cm, itemsep=0cm, parsep=0cm, label=\arabic*., after=\vspace{-0.1cm}, before=\vspace{-0.1cm}]
    \item Система должна быть написана на языке программирования Python.
    \item Система должна работать с файлами формата csv.
\end{enumerate}

\subsection{Архитектура приложения}
\label{subsec:Architecture}
В данном разделе представлена архитектура системы. Она представляет собой диаграмму компонентов, которая отображена на рисунке~\ref{ris:arch}.

\begin{figure}[ht]
    \center{\includegraphics[width=1\linewidth]{image}}
    \fbox{\includegraphics[scale=0.47]{Курсовая работа/pic/Архитектура системы.png}}
    \caption{Архитектура системы}
    \label{ris:arch}
\end{figure}


\textbf{Система для нахождения фиктивных аккаунтов} содержит модули, представленные ниже.

\textit{Головной модуль} отвечает за запуск остальных окон приложения, а также выводит подсказку для пользователя о том, как работать с системой.

\textit{Загрузка данных} отвечает за работу с файлами, которые Исследователь загружает в систему: открытие окна для выбора файла формата csv, добавление датасета в базу данных приложения.

\textit{Редактор} предоставляет функционал, который позволяет Исследователю изменять загруженные таблицы: модифицировать значения ячеек, добавлять и удалять столбцы и строки.

\textit{Настройки} позволяют Исследователю выбрать параметры алгоритмов, отображения результатов, сохранения файлов, а также найти <<фейки>>. Данный компонент передает значения в алгоритмы, вызывает модуль создания признаков.

\textit{Создание признаков} формирует новую таблицу, на основе выбранной пользователем. Он вычисляет необходимые для поиска фиктивных аккаунтов значения и загружает их в базу данных приложения с признаками.

\textit{DBSCAN} содержит алгоритм DBSCAN. Он получает значения из бызы данных с признаками, объединяет точки в кластеры, определяет аномалии и сохраняет их в csv-файл, вычисляет метрики на основе размеченных данных и создает визуализацию результатов.

\textit{Иерархическая кластеризация} содержит алгоритм агломеративной кластеризации и подбор количества оптимальных кластеров для нее. Модуль получает значения из базы данных с признаками, выполняет кластеризацию, находит аномальные группы и выводит их в csv-файл, вычисляет метрики и создает дендрограмму.

\textit{Дерево решений} содержит алгоритм дерева решений. Модуль загружает сохраненную модель дерева решений, классифицирует данные и сохраняет фиктивные аккаунты в csv-файл, вычисляет метрики и визуализирует дерево.

\textit{Случайный лес} содержит алгоритм случайного леса. Модуль загружает сохраненную модель случайного леса, создает предсказания на ее основе, определяет фиктивные аккаунты и сохраняет их в csv-файл, вычисляет метрики и визуализирует результаты работы.

\textit{Изоляционный лес} содержит алгоритм изоляционного леса. Компонент загружает сохраненную модель изоляционного леса, создает предсказания на ее основе, определяет аномалии и сохраняет их в csv-файл, вычисляет метрики, а также визуализирует результаты.

\textit{База данных} хранит загруженные пользователем датасеты, предоставляет их модулям редактирования и создания признаков.

\textit{База данных с признаками} хранит таблицы со значениями, вычисленными с помощью модуля создания признаков, предоставляет их алгоритмам.

\textit{Модели} -- это обученные и сохраненные модели, которые используются для нахождения фиктивных аккаунтов с помощью алгоритмов изоляционного леса, дерева решений и случайного леса.

\textbf{Обучение моделей} представлено описанными далее компонентами.

\textit{Аугментация} помогает сбалансировать набор данных, добавляя искусственно созданные записи на основе уже существующих. Результатом работы модуля является аугментированный датасет, содержащий равное количество фиктивных и настоящих аккаунтов. Компонент добавляет новые строки в конец таблицы и сохраняет ее в csv-файл. Также предоставляет функционал для создания таблицы с признаками.

\textit{Обучение моделей} -- модуль для обучения и сохранения моделей трех алгоритмов: изоляционного леса, дерева решений и случайного леса. 


\vspace{1.5em}
\subsection{Графический интерфейс}
\label{subsec:Graphic}
В данном разделе представлены макеты модулей системы. Они являются примерными и содержат в себе основной функционал.

На рисунке~\ref{ris:Main} представлен главный экран системы. На нем располагаются кнопки, по которым происходит работа с приложением. При нажатии на кнопку <<Загрузить данные>> открывается форма для выбора файла для загрузки, кнопка <<Отредактировать данные>> открывает новое окно, в котором можно выбрать таблицу и изменить ее. По кнопке <<Найти фейки>> открывается раздел, в котором можно ввести необходимые параметры алгоритмов и запустить их работу.

\begin{figure}[H]
    \center{\includegraphics[width=1\linewidth]{image}}
    \fbox{\includegraphics[scale=0.38]{Курсовая работа/pic/Главный экран.png}}
    \caption{Главный экран}
    \label{ris:Main}
\end{figure}

Экран с редактированием данных представлен на рисунке~\ref{ris:Razmetka}. На нем располагается таблица с данными, а также кнопки, которые позволяют взаимодействовать с ней: добавдлять и удалять строки и столбцы. Вверху экрана можно переключаться между загруженными датасетами с помощью выпадающего списка. Также на макете можно увидеть кнопку <<Сохранить>>, которая отвечает за сохранение изменений в данных.

\begin{figure}[h!]
    \center{\includegraphics[width=1\linewidth]{image}}
    \fbox{\includegraphics[scale=0.3]{Курсовая работа/pic/Разметка.png}}
    \caption{Экран разметки}
    \label{ris:Razmetka}
\end{figure}

На рисунке~\ref{ris:Settings01} показан экран с настройками. В нем представлен выбор алгоритма, таблицы с данными, необходимости визуализации результатов с помощью соответствующих выпадающих списков. Также на нем расположены текстовое поле и кнопка для выбора папки, в которую сохраняются файлы, созданные при работе алгоритмов, и кнопка для нахождения фиктивных аккаунтов. 

Помимо этого при выборе конкретного алгоритма будут изменяться виджеты, расположеннные в правой части экрана. Для DBSCAN появятся два текстовых поля с вводом $Eps$ и $MinPts$, при выборе изоляционного леса, дерева решений и случайного леса (пример на рисунке~\ref{ris:Settings02}) выпадающий список с выбором обученной модели, для иерархической кластеризации (рисунок~\ref{ris:Settings03}) отобразится выпадающий список с методами соединения.

\begin{figure}[H]
    \center{\includegraphics[width=1\linewidth]{image}}
    \fbox{\includegraphics[scale=0.35]{Курсовая работа/pic/Настройки_DBSCAN.png}}
    \caption{Выбор параметров DBSCAN}
    \label{ris:Settings01}
\end{figure}


\begin{figure}[H]
    \center{\includegraphics[width=1\linewidth]{image}}
    \fbox{\includegraphics[scale=0.35]{Курсовая работа/pic/Настройки_Isolation forest.png}}
    \caption{Выбор параметров изоляционного леса}
    \label{ris:Settings02}
\end{figure}

\begin{figure}[H]
    \center{\includegraphics[width=1\linewidth]{image}}
    \fbox{\includegraphics[scale=0.32]{Курсовая работа/pic/Настройки_иерархич класт.png}}
    \caption{Выбор параметров иерархической кластеризации}
    \label{ris:Settings03}
\end{figure}

\subsection{Базы данных}
\label{subsec:Graphic}

При использовании приложения создаются базы данных. Это происходит при загрузке новых файлов: создается база данных app\_database.db, а также перед работой алгоритмов для нахождения фейков: создается app\_database\_features.db, в которую добавляются таблицы (описание представлено в таблице~\ref{tabular:featuresDescription}) с числовыми характеристиками аккаунтов.

\begin{table}[H]
    \caption{Признаки аккаунтов}
    \label{tabular:featuresDescription}
    \vspace{1em}
    \small
    \setstretch{1.0}
    \begin{tabular}{|l|p{11cm}|}
    \hline
    \multicolumn{1}{|c|}{\textbf{Столбец}} & \multicolumn{1}{c|}{\textbf{Значение}}                                          \\ \hline
    user\_id                               & Уникальный идентификатор пользователя                                           \\ \hline
    symb\_in\_name                         & Доля небуквенных символов в имени пользователя                                                     \\ \hline
    symb\_in\_email                         & Доля небуквенных символов в адресе электронной почты                            \\ \hline
    time\_difference                       & Разница во времени между датой регистрации и последнего входа в аккаунт         \\ \hline
    neighbour\_above                       & Разница во времени с ближайшим зарегистрированным аккаунтом до текущего         \\ \hline
    neighbour\_below       & Разница во времени с ближайшим зарегистрированным аккаунтом после текущего            \\ \hline
    text\_neighbour\_above & Сходство текстовых полей текущего аккаунта и предыдущего зарегистрированного аккаунта \\ \hline
    text\_neighbour\_below & Сходство текстовых полей текущего аккаунта и следующего зарегистрированного аккаунта  \\ \hline
    is\_fake                               & Признак фальшивости аккаунта (1 -- если аккаунт фиктивный, 0 -- если настоящий) \\ \hline
    \end{tabular}
\end{table}


