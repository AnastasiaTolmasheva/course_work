\newpage
\section{Проектирование}
\label{sec:designing}
В разделе 2.1. описаны варианты использования системы. В разделе 2.2. представлены функциональные и нефункциональные требования к системе. Раздел 2.3 содержит модули системы. В разделе 2.4. показаны зарисовки графического интерфейса.

\vspace{1.5em}
\subsection{Варианты использования системы}
\label{subsec:Variants}
Описание способов взаимодействия с системой, кто с ней может работать и каким образом отображено на рисунке~\ref{ris:variantsUse}.

\begin{figure}[ht]
    \center{\includegraphics[width=1\linewidth]{image}}
    \includegraphics[scale=0.80]{Курсовая работа/pic/Варианты использования системы.png}
    \caption{Диаграмма вариантов использования}
    \label{ris:variantsUse}
\end{figure}

Единственный актер, который может взаимодействовать с системой распознавания фиктивных аккаунтов в Open Journal System -- это исследователь. Он может выполнять следующие действия:
\vspace{-1.5em}
\begin{enumerate}[itemsep=0pt, topsep=1.5em]
    \item Загрузить данные. Исследователь совершает выгрузку данных аккаунтов.
    \item Редактировать данные. Исследователь может отредактировать данные, которые загрузил в систему.
    \item Выполнить аугментацию. Исследователь может искусственно расширить имеющийся набор данных, сгенерировав новые, и привести выборку к равному количеству фиктивных и подлинных аккаунтов.
    \item Выполнить настройку. Исследователь выполняет настройку того, какой будет выполняться алгоритм, с какими параметрами, в каком формате будут выводиться результаты.
    \item Обучить модель. Исследователь может обучить модель.
    \item Найти фиктивные аккаунты. Исследователь может найти <<фейки>> с помощью выбранного алгоритма.
\end{enumerate}
\vspace{-1.5em}

Спецификация вариантов использования системы представлена в Приложении Б.

\vspace{1.5em}
\subsection{Требования к системе}
\label{subsec:fotmalDefinition}
\textbf{Функциональные требования}

Функциональные требования -- это спецификация того, каким образом должно вести себя разрабатываемое программное обеспечение. Они определяют, какие конкретные функции и возможности должны быть включены в приложение, чтобы оно соответствовало ожиданиям пользователей. Эти требования описывают конкретные действия, которые система должна выполнять, и каким образом пользователь взаимодействует с приложением для достижения определенных целей. 

Функциональные требования к системе:
\vspace{-1.5em}
\begin{enumerate}[itemsep=0pt, topsep=1.5em]
    \item Система должна предоставлять пользователю возможность загружать данные для исследования.
    \item Система должна предоставлять пользователю возможность выполнить настройку ее работы: выбрать алгоритм, изменить его характеристики, добавить вывод результатов в графической форме.
    \item Система должна предоставлять возможность обучать модель.
    \item Система должна находить фиктивные аккаунты.
\end{enumerate}
\vspace{-1.5em}

\textbf{Нефункциональные требования}

Ненфункциональные требования определяют свойства и характеристики, которыми должна обладать система. Эти требования касаются аспектов, не связанных напрямую с конкретными функциями, а определяют общие качественные аспекты, которые необходимы для функционирования системы.

Нефункциональные требования к системе:
\vspace{-1.5em}
\begin{enumerate}[itemsep=0pt, topsep=1.5em]
    \item Система должна быть написана на языке программирования Python.
    \item Система должна работать с файлами формата csv.
\end{enumerate}

\subsection{Архитектура приложения}
\label{subsec:Architecture}
В данном разделе представлена архитектура системы. Она представляет из себя диаграмму компонентов, которая представлена на рисунке~\ref{ris:arch}.

\begin{figure}[ht]
    \center{\includegraphics[width=1\linewidth]{image}}
    \includegraphics[scale=0.65]{Курсовая работа/pic/Архитектура системы.png}
    \caption{Архитектура системы}
    \label{ris:arch}
\end{figure}

Модули, представненные на схеме:

\textit{Модуль загрузки данных} отвечает за работу с файлами, которые исследователь загружает в систему: открытие окна для выбора файла формата CSV, добавление его в базу данных приложения.


\textit{Модуль редактирования} предоставляет функционал, который позволяет исследователю редактировать загруженный датасет: изменять значение ячеек, добавлять и удалять столбцы и строки.


\textit{Модуль аугментации} помогает исследователю сбалансировать набор данных, добавляя искусственно созданные записи на основе уже существующих. Результатом работы модуля является аугментированный датасет, содержащий равное количество фиктивных и настоящих аккаунтов. Компонент добавляет новые строки в конец таблицы и загружает ее в базу данных приложения.


\textit{Модуль настройки} позволяет исследователю выполнить настройку алгоритмов, отображения результатов, сохранения файлов, а также найти <<фейки>>. Данный компонент передает параметры в алгоритмы, вызывает модули создания признаков и выполняения аугментации.


\textit{Модуль создания признаков} создает новую таблицу, на основе выбранной пользователем, которая форматирует текстовые данные датасета в числовые. Данная таблица загружается в новую базу данных.


\textit{Модуль алгоритма DBSCAN} содержит алгоритм DBSCAN для нахождения аномалий в виде фиктивных аккаунтов, его визуализацию и вычисление метрик.


\textit{Модуль алгоритма иерархической кластеризации} содержит алгоритм агломеративной кластеризации, подбор количества оптимальных кластеров для нее, визуализацию работы алгоритма (создание дендрограммы), а также вычисление метрик.


\textit{Главный модуль} показывает исследователю, как работать с приложением, а также отвечает за запуск остальных окон приложения.


\vspace{1.5em}
\subsection{Графический интерфейс}
\label{subsec:Graphic}
В данном разделе представлены макеты модулей системы. Они являются примерными и содержат в себе основной функционал.

На рисунке~\ref{ris:Main} представлен главный экран системы. На нем располагаются кнопки, по которым происходит работа с приложением. При нажатии на кнопку <<Загрузить данные>> открывается форма для выбора файла для загрузки, кнопка <<Отредактировать данные>> открывает новое окно, в котором можно выбрать таблицу для редактирования. По кнопке <<Найти фейки>> открывается раздел, в котором можно выбрать необходимые параметры алгоритмов и запустить их работу.

\begin{figure}[H]
    \center{\includegraphics[width=1\linewidth]{image}}
    \includegraphics[scale=0.5]{Курсовая работа/pic/Главный экран.png}
    \caption{Главный экран}
    \label{ris:Main}
\end{figure}

Экран с редактированием данных представлен на рисунке~\ref{ris:Razmetka}. На нем располагается таблица с данными, а также кнопки, которые позволяют взаимодействовать с ней. Вверху экрана можно переключиться между загруженными данными, если пользователь добавлял несколько файлов. Также на макете можно увидеть кнопку <<Сохранить>>, которая отвечает за сохранение изменений в данных.

\begin{figure}[H]
    \center{\includegraphics[width=1\linewidth]{image}}
    \includegraphics[scale=0.3]{Курсовая работа/pic/Разметка.png}
    \caption{Экран разметки}
    \label{ris:Razmetka}
\end{figure}

На рисунке~\ref{ris:Settings0} показан экран с настройками. На нем можно выбрать алгоритм, таблицу с данными, отображать результаты или нет, а также попку для вывода результатов. Помимо этого после выбора конкретного алгоритма будут появляться новые поля для настройки конкретного алгоритма (пример представлен на рисунке~\ref{ris:Settings01}).

\begin{figure}[H]
    \center{\includegraphics[width=1\linewidth]{image}}
    \includegraphics[scale=0.3]{Курсовая работа/pic/Настройки_общие.png}
    \caption{Экран настроек}
    \label{ris:Settings0}
\end{figure}

\begin{figure}[H]
    \center{\includegraphics[width=1\linewidth]{image}}
    \includegraphics[scale=0.3]{Курсовая работа/pic/Настройки_DBSCAN.png}
    \caption{Пример появления новых полей при выборе алгоритма}
    \label{ris:Settings01}
\end{figure}

\vspace{1.5em}
\subsection{Базы данных}
\label{subsec:Graphic}

Для того, чтобы пользователь мог загружать данные, при использовании приложения создаются базы данных. Заранее нельзя определить, какие таблицы с какими колонками будут загружены, но для того, чтобы исследования проходили корректно, будут выводиться предупредления об отсутствии основных столбцов в таблице (user\_id, username, email, country, date\_registered, date\_last\_login), так как в основном именно эти данные преобразовываются в числовые значения. 

Кроме того, нельзя заранее сказать, какие могут содержаться таблицы в базах данных, поэтому база данных app\_database.db создается только при загрузке каких-либо датасетов (пример таблицы: в разделе~\ref{subsec:Dataset}<<Набор данных для обучения>> описана таблица, с которой происходила работа при создании системы). 
База данных app\_database\_features.db создается в определенный момент при настройке алгоритмов, она не видна пользователю, он не может взаимодействовать с ней. Она содержит в себе преобразованные в числовые значения параметры выбранного датасета. Столбцы и описание данных, созданной на основе набора данных для обучения таблицы, представлены в таблице~\ref{subsec:featuresDescription} 

