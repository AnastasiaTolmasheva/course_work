% Formatting of listings
\lstset{language=C, frame=L, basicstyle=\footnotesize,%\sffamily,
	keywordstyle=\bfseries, showstringspaces=false, xleftmargin=\parindent, numbers=none, numberstyle=\tiny, stepnumber=2, numbersep=5pt}
\newpage
\section{Реализация}

\subsection{Архитектура приложения}
\label{subsec:Architecture}
Показать модули приложения, расписать то, что делает каждый из них.

\vspace{1.5em}
\label{sec:Realisation}
\subsection{Программные средства реализации}
Прописать используемые языки программирования и библиотеки.

\vspace{1.5em}
\subsection{Подготовка данных}
\label{subsec:Preparation}
Сбор данных о пользователях системы электронного журнал, определение, какие аккаунты являются фиктивными.

\vspace{1.5em}
\subsection{Инженерия признаков}
\label{subsec:Characteristics}
У фиктивных аккаунтов есть некоторые признаки, которые могут отличать их от обычных аккаунтов. Далее рассмотрим их подробнее.
\begin{itemize}
    \item user\_id: уникальный идентификатор пользователя. Используется для идентификации каждого аккаунта;
    \item username\_length: длина имени пользователя;
    \item numbers\_in\_name: переменная, которая означает наличие цифр в имени пользователя. Если нет – ставится 0, иначе – 1;
    \item email\_length: длина email;
    \item matching\_names: переменная, которая означает совпадение username с email по определенному порогу сходства (в случае несовпадения 0, иначе – 1);
    \item pattern\_email: продходит ли email по шаблону user@domain.com (в случае прохождения по парамтру 0, иначе – 1);
    \item country: проверка, указана ли страна (если не указана - 0, иначе - 1);
    \item date\_last\_email: проверка даты последнего отправленного email. Если она есть – 1, иначе – 0;
    \item date\_registered: дата регистрации аккаунта;
    \item date\_last\_login: дата последнего входа в аккаунт;
    \item matching\_dates: характеристика, совпадают ли даты послегнего входа в аккаунт и даты регистрации.
    \item username\_neighbour\_above: расстояние до <<соседа>> по базе данных сверху рассматриваемого аккаунта, который имеет сходные символы в username.
    \item username\_neighbour\_below: расстояние до <<соседа>> по базе данных снизу рассматриваемого аккаунта, который имеет сходные символы в username.
    \item email\_neighbour\_above: расстояние до <<соседа>> по базе данных сверху рассматриваемого аккаунта, который имеет сходные символы в логине email.
    \item email\_neighbour\_below: расстояние до <<соседа>> по базе данных снизу рассматриваемого аккаунта, который имеет сходные символы в логине email.
\end{itemize}

\subsection{Реализация компонентов приложения}
\label{subsec:Components}
Расписать работу модулей приложения, какие данные получают на входе, какие на выходе.

\vspace{1.5em}
\subsection{Реализация пользовательского интерфейса}
\label{subsec:UserInt}
Каким образом и где реализован интерфейс приложения. Описать основные механизмы и привести изображения.