% Formatting of listings
\lstset{language=C, frame=L, basicstyle=\footnotesize,%\sffamily,
	keywordstyle=\bfseries, showstringspaces=false, xleftmargin=\parindent, numbers=none, numberstyle=\tiny, stepnumber=2, numbersep=5pt}
\newpage
\section{Реализация}
%В разделе 3.1 описаны средства реализации: выбранный язык программирования, редактор кода и подключаемые библиотеки. В разделе 3.2 описана реализация компонентов приложения. В разделе 3.3 показана реализация пользовательского интерфейса с изображениями окон. Описание обучения моделей расположено в разделе 3.4.

%\vspace{1.5em}
\label{sec:Realisation}
\subsection{Программные средства реализации}
Система нахождения фиктивных аккаунтов была написана на языке программирования Python версии 3.9.13. Разработка велась в текстовом редакторе Visual Studio Code 1.89.0. Основные библиотеки, которые использовались при реализации:

\begin{enumerate}[itemindent=2cm, leftmargin=0cm, labelsep=0.3cm, topsep=0cm, itemsep=0cm, parsep=0cm, label=\arabic*., after=\vspace{-0.1cm}, before=\vspace{-0.1cm}]
    \item Tkinter~\cite{tkinter} -- библиотека для создания графических пользовательских интерфейсов в Python.
    \item Sqlite3~\cite{sqlite3} -- библиотека для работы с базой данных SQLite. Позволяет создавать, читать, изменять и удалять записи в реляционных базах данных, используя SQL-запросы.
    \item Pandas~\cite{pandas} -- библиотека для анализа и обработки данных в Python.
    \item Numpy~\cite{numpy} -- библиотека для работы с массивами и матрицами.
    \item Scikit-learn~\cite{scikit-learn} -- библиотека для машинного обучения и анализа данных.
    \item Matplotlib.pyplot~\cite{maptolib} -- подбиблиотека Matplotlib, позволяющая создавать графики и визуализировать данные.
\end{enumerate}
\vspace{1.5em}

\subsection{Реализация компонентов приложения}
\label{subsec:Components}
В данном разделе приводится описание реализации компонентов системы. Полный код представлен в репозитории на github~\cite{git}.

\textbf{Реализация модуля загрузки}

Модуль загрузки предназначен для создания таблиц в базе данных и загрузки в нее датасетов. Он имеет две соответствующие функции:

\begin{itemize}[itemindent=2cm, leftmargin=0cm, labelsep=0.3cm, topsep=0cm, itemsep=0cm, parsep=0cm, before=\vspace{-0.15cm}, after=\vspace{-0.15cm}]
    \item Create\_table: на вход принимает table\_name -- строковая переменная с названим таблицы и columns -- список столбцов. Подключается к базе данных app\_database, в которой создает пустую таблицу по переданному шаблону.
    \item Loading: позволяет выбрать csv-файл с помощью диалогового окна. Функция проверяет, существует ли таблица с таким же названием в базе данных. В случае, если она уже есть, предлагает Исследователю перезаписать данные. Она считывает название файла и заголовки столбцов, вызывает функцию create\_table и вставляет данные в новую таблицу.
\end{itemize}


\textbf{Реализация модуля создания признаков}

Модуль создает таблицу с характеристиками на основе данных из исходного датасета. 

\begin{itemize}[itemindent=2cm, leftmargin=0cm, labelsep=0.3cm, topsep=0cm, itemsep=0cm, parsep=0cm, before=\vspace{-0.15cm}, after=\vspace{-0.15cm}]
    \item Find\_time\_neighbours: на вход принимает dates -- колонку с датами регистрации (имеющую тип series -- объект библиотеки pandas), возвращает четыре списка: с разницами во времени в секундах до ближайшего зарегистрированного аккаунта до и после каждой записи, а также индексы этих соседних аккаунтов. Внутри функции используется цикл, который для каждой даты регистрации вычисляет разницу во времени с остальными датами и записывает их в два списка: положительные и отрицательные значения. После этого ищется наибольшее отрицательное число и наименьшее положительное число, которые записываются в итоговые списки, а индексы таких записей сохраняются. Если ближайших дат до или после текущей не было найдено, то записывается \texttt{NULL}.
    \item Harmonic\_mean: на вход принимает два целых числа, возвращает их среднее гармоническое (число с плавающей точкой). Если сумма входных данных равна нулю, то на выход подается 0.
    \item Calculate\_damerau\_levenshtein\_distance: на входе получает indexes -- индексы соседних аккаунтов и data -- датафрейм с исходными данными (таблица, объединение столбцов pandas.Series). Возвращает список distances. Функция с помощью цикла вычисляет расстояние Дамерау-Левенштейна~\cite{Damerau64} между именами пользователей и электронными почтами ближайших зарегистрированных аккаунтов, а после вычисляет их среднее гармоническое, которое записывает в список.
    \item Making\_features: принимает строковую переменную с именем таблицы. Осуществляется чтение данных из нее, после чего вычисляются характеристики, описанные в разделе~\ref{subsec:Graphic} в таблице~\ref{tabular:featuresDescription}. Данные нормализиются по столбцам, пустые значения заполняются нулями. Затем функция подключается к новой базе данных и создает таблицу в ней, куда записывает рассчитанные признаки.
\end{itemize}

\textbf{Реализация DBSCAN}

Модуль содержит функцию run\_dbscan\_algorithm, которая предназначена для нахождения аномалий. На вход она принимает table\_name -- строковая переменная с названием таблицы, eps (радиус окрестности, имеет тип числа с плавающей точкой), min\_samples (минимальное количество точек в окрестности, целое число), visualization -- строка с флагом в необходимости визуализации и folder -- строка, содержащая путь к папке для сохранения результатов.

Работа происходит следующим образом:

\begin{enumerate}[itemindent=2cm, leftmargin=0cm, labelsep=0.3cm, topsep=0cm, itemsep=0cm, parsep=0cm, label=\arabic*., after=\vspace{-0.1cm}, before=\vspace{-0.1cm}]
    \item Подготовка данных. Устанавливается соединение с базой данных признаков (app\_database\_features.db), выполняется запрос, чтобы извлечь данные из указанной таблицы. Колонка user\_id становится индексом, создается копия датафрейма, а из изначальной таблицы удаляется столбец is\_fake с метками. 
    \item Применение алгоритма DBSCAN с заданными параметрами eps и min\_samples. Результаты кластеризации сохраняются в колонке cluster.
    \item Определение аномалий. После кластеризации шум помечается значением 1 (как фиктивные аккаунты в столбце is\_fake), а все остальные точки получают метку 0 (как нормальные аккаунты в столбце is\_fake).
    \item Сохранение аномалий и их данных в csv-файл.
    \item Вычисление метрик и их вывод в csv-файл. В качестве меток используется колонка is\_fake в копии датафрейма.
    \item Визуализация результатов. Если Исследователь пожелал, создается визуализация. На графике отображаются точки данных, а также выводятся значения вычисленных метрик.
\end{enumerate}


\textbf{Реализация иерархической кластеризации}

Модуль содержит функции для выполнения агломеративной кластеризации и поиска оптимального числа кластеров для нее.

\begin{itemize}[itemindent=2cm, leftmargin=0cm, labelsep=0.3cm, topsep=0cm, itemsep=0cm, parsep=0cm, before=\vspace{-0.15cm}, after=\vspace{-0.15cm}]
    \item Find\_optimal\_clusters: принимает data -- датафрейм с признаками, linkage\_method -- строка с методом соединения, возвращает optimal\_clusters -- оптимальное число кластеров для алгоритма и Z -- многомерный массив, матрица расстояний для построения дендрограммы. Функция объединяет точки в кластеры, высчитывая расстояния между ними. Для определения оптимального количества кластеров был выбран порог расстояния (70\% от максимального). Вычисляется количество кластеров, у которых расстояние превышает заданный порог.
    \item Run\_hierarchical\_clustering: на вход получает строки с названием выбранной таблицы, методом соединения, флагом в необходимости визуализации и директорией для сохранения результатов. Данные извлекаются из базы данных с признаками. Колонка user\_id становится индексом, создается копия датафрейма, а из изначальной таблицы удаляется столбец is\_fake. Определяется оптимальное количество кластеров с использованием функции find\_optimal\_clusters и применяется агломеративная кластеризация. Находятся аномалии: если размер кластера меньше одной пятой максимального размера кластеров, то считаем его выделяющимся. Эти данные записываются в csv-файл. Метрики вычисляются и также сохраняются в файл. Если требуется, создается дендрограмма.
\end{itemize}


\textbf{Реализация изоляционного леса}

Модуль содержит функцию run\_isolation\_forest\_algorithm, которая принимает строки с названием таблицы, именем модели изоляционного леса, флагом в необходимости визуализации и папкой для сохранения результатов. Она предназначена для нахождения аномалий в виде фиктивных аккаунтов.

Порядок действий функции run\_isolation\_forest\_algorithm:

\begin{enumerate}[itemindent=2cm, leftmargin=0cm, labelsep=0.3cm, topsep=0cm, itemsep=0cm, parsep=0cm, label=\arabic*., after=\vspace{-0.1cm}, before=\vspace{-0.1cm}]
    \item Загружается выбранная Исследователем модель.
    \item Данные извлекаются из таблицы базы данных с признаками. Колонка user\_id становится индексом, создается копия датафрейма, а из изначальной таблицы удаляется столбец is\_fake.
    \item Проводятся предсказания на основе выбранной модели, которые преобразуются к меткам 1 и 0 (как в колонке is\_fake).
    \item Аномалии с метками 1 сохраняются в csv-файл.
    \item Вычисляются метрики, в качестве меток используется колонка is\_fake скопированного датафрейма. После этого они записываются в отдельный csv-файл.
    \item Если Исследователь пожелал, создается визуализация. На графике отображаются точки данных, выводятся значения вычисленных метрик.
\end{enumerate}


\vspace{-1.5em}
\textbf{Реализация дерева решений}

Модуль содержит функцию run\_decision\_tree\_algorithm, которая принимает название таблицы, имя модели дерева решений, флаг с необходимостью визуализации и путь к папке для сохранения результатов. Она классифицирует данные и выводит информацию о найденных фиктивных аккаунтах.

Порядок действий функции run\_decision\_tree\_algorithm:

\begin{enumerate}[itemindent=2cm, leftmargin=0cm, labelsep=0.3cm, topsep=0cm, itemsep=0cm, parsep=0cm, label=\arabic*., after=\vspace{-0.1cm}, before=\vspace{-0.1cm}]
    \item Загружается выбранная Исследователем модель.
    \item Данные извлекаются из таблицы базы данных с признаками. Колонка user\_id становится индексом, датасет делится на признаки и целевую переменную.
    \item Проводятся предсказания на основе выбранной модели.
    \item Данные найденных фиктивных аккаунтов выводятся в csv-файл.
    \item Вычисляются метрики для оценки работы модели. Они сохраняются в отдельный файл.
    \item Если Исследователь пожелал, создается визуализация дерева решений.
\end{enumerate}


\textbf{Реализация случайного леса}

Модуль содержит функцию run\_random\_forest\_algorithm, которая принимает название таблицы, имя модели случайного леса, флаг с необходимостью визуализации и путь к директории для сохранения результатов. Данная функия делает то же самое, что и run\_decision\_tree\_algorithm (классификация, вывод результатов в файл), но с одним отличием -- вместо визуализации дерева решений созадется график с точками.


\vspace{1.5em}
\subsection{Реализация пользовательского интерфейса}
\label{subsec:UserInt}
Реализация пользовательского интерфейса проводилась по разработанным макетам. Было создано три основных окна: главное меню, окно редактирования данных и окно настроек алгоритмов. Для реализации использовалась библиотека tkinter~\cite{tkinter}.

На рисунке~\ref{ris:main_window} изображено главное окно приложения. На нем представлено меню с тремя кнопками: <<Загрузить данные>>, <<Отредактировать данные>>, <<Найти фейки>>. При нажатии на первую вызывается функция loading из модуля загрузки данных, появляется диалоговое окно для выбора файла. По кнопке <<Отредактировать данные>> открывается экран, который позволяет изменять таблицы. Кнопка <<Найти фейки>> отвечает за открытие окна с настройками алгоритмов и поиском фиктивных аккаунтов. В правом нижнем углу главного окна расположена кнопка с вопросительным знаком, при нажатии на которую появляется окно с информацией о системе (рисунок~\ref{ris:help}).


\begin{figure}[h!]
    \center{\includegraphics[width=1\linewidth]{image}}
    \fbox{\includegraphics[scale=0.40]{Курсовая работа/pic/главное окно.png}}
    \caption{Главное окно приложения}
    \label{ris:main_window}
\end{figure}


\begin{figure}[H]
    \center{\includegraphics[width=1\linewidth]{image}}
    \fbox{\includegraphics[scale=0.40]{Курсовая работа/pic/Помощь.png}}
    \caption{Всплывающее окно с информацией для пользователя}
    \label{ris:help}
\end{figure}


На рисунке~\ref{ris:edit_window} представлено окно для редактирования данных. Вверху расположен выпадающий список, который позволяет переключаться между загруженными в базу данных таблицами. Данные выбранного датасета отображаются в иерархическом списке, который можно пролистывать с помощью вертикального и горизонтального скроллеров. 

Внизу окна находятся кнопки <<Добавить строку>>, <<Удалить строку>>, <<Добавить столбец>>, <<Удалить столбец>>, они отвечают за изменение структуры таблицы. Строка автоматически добавляется в конец иерархического списка, удалить ее можно при выделении и нажатии на кнопку. Для взаимодействия со столбцами открываются диалоговые окна, в которые нужно ввести значение добавляемого или удаляемого столбца. Кнопка <<Сохранить>> применяет внесенные в датасет изменения. 

\begin{figure}[H]
    \center{\includegraphics[width=1\linewidth]{image}}
    \fbox{\includegraphics[scale=0.24]{Курсовая работа/pic/редактор данных.png}}
    \caption{Окно редактирования данных}
    \label{ris:edit_window}
\end{figure}


При двойном клике на ячейку таблицы появляется окно с возможностью редактирования ее значения (рисунок~\ref{ris:edit_cell}). На нем расположено текстовое поле для ввода данных, а также две кнопки, позволяющие применить или отклонить внесенные изменения.


\begin{figure}[H]
    \center{\includegraphics[width=1\linewidth]{image}}
    \fbox{\includegraphics[scale=0.8]{Курсовая работа/pic/редактирование_ячейки.png}}
    \caption{Изменение значения ячейки}
    \label{ris:edit_cell}
\end{figure}


На рисунке~\ref{ris:settings_window} представлено окно с настройками алгоритмов. На нем расположено несколько выпадающих списков: для выбора алгоритма, таблицы с данными, а также необходимости визуализации. При нажатии на кнопку <<Выбрать>> открывается диалоговое окно, в котором исследователь может подобрать папку для сохранения результатов работы алгоритма. Путь к ней отображается в соседнем текстовом поле. Внизу расположена кнопка <<Найти фейки>>, которая запускает работу алгоритмов.

Кроме того, при выборе определенного алгоритма меняются виджеты в правой части окна. Для изоляционного леса, дерева решений и случайного леса отображаются выпадающий список с выбором модели, для DBSCAN (рисунок~\ref{ris:settings_window_01}) два поля для ввода его параметров, а для иерархической кластеризации (рисунок~\ref{ris:settings_window_02}) выпадающий список с выбором метода соединения.


\begin{figure}[H]
    \center{\includegraphics[width=1\linewidth]{image}}
    \fbox{\includegraphics[scale=0.25]{Курсовая работа/pic/Наастройки_изоляционный лес.png}}
    \caption{Окно настроек: изоляционный лес}
    \label{ris:settings_window}
\end{figure}


\begin{figure}[H]
    \center{\includegraphics[width=1\linewidth]{image}}
    \fbox{\includegraphics[scale=0.25]{Курсовая работа/pic/Настройки окно 1.png}}
    \caption{Окно настроек: DBSCAN}
    \label{ris:settings_window_01}
\end{figure}


\begin{figure}[H]
    \center{\includegraphics[width=1\linewidth]{image}}
    \fbox{\includegraphics[scale=0.30]{Курсовая работа/pic/Настройки окно 2.png}}
    \caption{Окно настроек: иерархическая кластеризация}
    \label{ris:settings_window_02}
\end{figure}


При выполнении какой-либо задачи пользователь получает уведомление от системы о результате работы. В случае, когда не были выбраны данные или они были введены некорректно, появляется высплавающее окно с информацией о предупреждении (пример представлен на рисунке~\ref{ris:warning}).


\begin{figure}[H]
    \center{\includegraphics[width=1\linewidth]{image}}
    \fbox{\includegraphics[scale=0.5]{Курсовая работа/pic/предупреждение.png}}
    \caption{Предупреждение об ошибке}
    \label{ris:warning}
\end{figure}

Если данные введены корректно или операция прошла успешно, приложение оповещает пользователя соответствующим уведомлением (пример на рисунке~\ref{ris:info}).

\begin{figure}[H]
    \center{\includegraphics[width=1\linewidth]{image}}
    \fbox{\includegraphics[scale=0.6]{Курсовая работа/pic/Сообщение.png}}
    \caption{Диаграмма взаимодействия Исследователя с системой}
    \label{ris:info}
\end{figure}


На диаграмме последовательности (рисунок~\ref{ris:diag_posl}) представлена работа Исследователя с системой, а именно процесс настройки алгоритмов и в каких случая пользователь получает сообщения от приложения.


\begin{figure}[H]
    \center{\includegraphics[width=1\linewidth]{image}}
    \fbox{\includegraphics[scale=0.49]{Курсовая работа/pic/Диаграмма последовательности.png}}
    \caption{Диаграмма взаимодействия Исследователя с системой}
    \label{ris:diag_posl}
\end{figure}


\label{sec:Realisation}
\subsection{Реализация обучения моделей}

\textbf{Реализация аугментации}

Модуль аугментации необходим для расширения датасета. Данный компонент предоставляет возможность искусственно увеличивать количество записей фиктивных или настоящих аккаунтов, уравнивая их число. Также он позволяет создавать csv файлы с числовыми характеристиками аккаунтов.

\begin{itemize}[itemindent=2cm, leftmargin=0cm, labelsep=0.3cm, topsep=0cm, itemsep=0cm, parsep=0cm, before=\vspace{-0.15cm}, after=\vspace{-0.15cm}]
    \item Save\_data\_to\_csv: на вход получает датафрейм с данными, строки с именем таблицы, и путем до корня приложения. Функция сохраняет данные в csv файл.
    \item Get\_max\_user\_id: принимает датафрейм. Возвращает максимальное значение user\_id либо ничего, если нет такого столбца. 
    \item Balance\_data: функция принимает строку с названием файла, данные которого используются для аугментации. Выводит количество фиктивных и настоящих аккаунтов до и после аугментации. Данные из csv файла считываются, находится количество фиктивных и настоящих аккаунтов, и определяется, какие записи необходимо создавать. Get\_max\_user\_id находит максимальный ID пользователя, чтобы новые записи были расположены в порядке увеличения user\_id в конце датасета. Далее функция создает записи на основе выбора случайных значений из списков с данными столбцов. После этого объединяет аугментированный и обычный датафреймы, вызывает функцию save\_data\_to\_csv для сохранения полной таблицы в файл.
    \item Making\_features: принимает имя файла с данными. Создает характеристики, аналогичные тем, что создает модуль создания признаков (подробнее в разделе~\ref{subsec:Components} Реализация компонентов приложения), но сохраняет их в csv файл.
\end{itemize}


\textbf{Реализация модуля обучения моделей}

Данный компонент необходим для обучения моделей изоляционного леса, дерева решений и случайного леса. Кроме того, он вызывает функции из модуля аугментации для создания таблиц с признаками. 

В нем содержится функция teaching\_models. На вход она принимает строки с именем файла, на основе которого модель обучается, названием алгоритма, а также список словарей с параметрами. 
    
Данные считываются, делятся на признаки и метки. В зависимости от выбранного алгоритма функция выбирает класс модели. Для каждого переданного набора параметров создается и обучается модель, которая сохраняется в отдельную папку в корне приложения.
