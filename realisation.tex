% Formatting of listings
\lstset{language=C, frame=L, basicstyle=\footnotesize,%\sffamily,
	keywordstyle=\bfseries, showstringspaces=false, xleftmargin=\parindent, numbers=none, numberstyle=\tiny, stepnumber=2, numbersep=5pt}
\newpage
\section{Реализация}
\label{sec:Realisation}
\subsection{Программные средства реализации}
\label{subsec:Recourses}
Прописать используемые языки программирования и библиотеки.

\vspace{1.5em}
\subsection{Подготовка данных}
\label{subsec:Preparation}
Сбор данных о пользователях системы электронного журнал, определение, какие аккаунты являются фиктивными.

\vspace{1.5em}
\subsection{Инженерия признаков}
\label{subsec:Characteristics}
У фиктивных аккаунтов есть некоторые признаки, которые могут отличать их от обычных аккаунтов. Далее рассмотрим их подробнее.
\begin{enumerate}
    \item Недавняя дата регистрации (по исследованию [4] 15.80\% фиктивных аккаунтов и 2.80\% настоящих аккаунтов имели регистрацию, совершенную в течение прошедшего месяца)
    \item Дата последнего визита страницы (в основном страницу после со-здания не посещают)
    \item Статус страницы (удалена, заблокирована)
    \item Отсутствие логина или его части в адресе почты
    \item Пользователь должен сменить пароль (что это за столбец)
\end{enumerate}

\vspace{1.5em}
\subsection{Реализация компонентов приложения}
\label{subsec:Components}
Расписать работу модулей приложения, какие данные получают на входе, какие на выходе.

\vspace{1.5em}
\subsection{Реализация пользовательского интерфейса}
\label{subsec:UserInt}
Каким образом и где реализован интерфейс приложения. Описать основные механизмы и привести изображения.