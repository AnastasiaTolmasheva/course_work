% Formatting of listings
\lstset{language=C, frame=L, basicstyle=\footnotesize,%\sffamily,
	keywordstyle=\bfseries, showstringspaces=false, xleftmargin=\parindent, numbers=none, numberstyle=\tiny, stepnumber=2, numbersep=5pt}
\newpage
\section{Реализация}
В разделе 3.1 описаны средства реализации: выбранный язык программирования, редактор кода и подключаемые библиотеки. В разделе 3.2 описана реализация компонентов приложения (какие данные получают модули, как их преобразуют). В разделе 3.2 показана реализация пользовательского интерфейса с изображениями окон.

\vspace{1.5em}
\label{sec:Realisation}
\subsection{Программные средства реализации}
Система нахождения фиктивных аккаунтов была написана на языке программирования Python версии 3.9.13. Разработка велась в текстовом редакторе Visual Studio Code 1.89.0. Основные библиотеки, которые использовались при реализации:

\vspace{-1.5em}
\begin{enumerate}[itemsep=0pt, topsep=1.5em]
    \item Tkinter -- библиотека для создания графических пользовательских интерфейсов в Python~\cite{tkinter}.
    \item Sqlite3 -- библиотека для работы с базой данных SQLite. Позволяет создавать, читать, изменять и удалять записи в реляционных базах данных, используя SQL-запросы~\cite{sqlite3}.
    \item Pandas -- библиотека для анализа и обработки данных в Python~\cite{pandas}.
    \item Scikit-learn -- библиотека для машинного обучения и анализа данных~\cite{scikit-learn}.
    \item Matplotlib.pyplot -- подбиблиотека Matplotlib, позволяющая создавать графики и визуализировать данные~\cite{maptolib}.
    \item Numpy -- библиотека для работы с массивами и матрицами~\cite{numpy}.
\end{enumerate}

\subsection{Реализация компонентов приложения}
\label{subsec:Components}
Расписать работу модулей приложения, какие данные получают на входе, какие на выходе.

\vspace{1.5em}
\subsection{Реализация пользовательского интерфейса}
\label{subsec:UserInt}
Реализация пользовательского интерфейса проводилась по разработанным макетам. Было создано три окна: главное меню, оекно редактирования данных и окно настроек алгоритмов. Для реализации использовалась библиотека tkinter~\cite{tkinter}.

На рисунке~\ref{ris:main_window}

\begin{figure}[!ht]
    \center{\includegraphics[width=1\linewidth]{image}}
    \includegraphics[scale=0.57]{Курсовая работа/pic/главное окно.png}
    \caption{Кластеризация методом изолированного леса}
    \label{ris:main_window}
\end{figure}

\begin{figure}[!ht]
    \center{\includegraphics[width=1\linewidth]{image}}
    \includegraphics[scale=0.24]{Курсовая работа/pic/редактор данных.png}
    \caption{Кластеризация методом изолированного леса}
    \label{ris:edit_window}
\end{figure}

\begin{figure}[!ht]
    \center{\includegraphics[width=1\linewidth]{image}}
    \includegraphics[scale=0.4]{Курсовая работа/pic/Настройки окно.png}
    \caption{Кластеризация методом изолированного леса}
    \label{ris:settings_window}
\end{figure}